\documentclass[a4paper, 12pt]{article}

\usepackage[utf8]{inputenc}
\usepackage[french]{babel}
\usepackage[T1]{fontenc}
\usepackage{geometry}
\usepackage{amsmath}
\usepackage{amsfonts}
\usepackage{amssymb}
\usepackage{adjustbox}
\usepackage{rotating}
\usepackage{graphicx}
\usepackage{wrapfig}
\usepackage{lscape}
\usepackage{epstopdf}
\usepackage{pdflscape}
\usepackage{titles}
\usepackage{float}
\usepackage{uniquecounter}
\usepackage{hyperref}
\usepackage{enumitem}
\usepackage[table]{xcolor}
\usepackage{array}
\usepackage{bookmark}
\usepackage{booktabs}
\usepackage{pifont}
\usepackage{colortbl}
\usepackage{multirow}
\usepackage{longtable}
\usepackage{adjustbox}
\usepackage{ltablex, array}
\usepackage{tikz}
\usepackage{fontawesome5}
\usepackage{utfsym}
\usepackage{fourier}
\usepackage{pdfpages}
\usepackage{lscape}
\usepackage{tabularx}

\hypersetup{
    colorlinks,
    citecolor=black,
    filecolor=black,
    linkcolor=black,
    urlcolor=black
}

\definecolor{grey_prestation}{RGB}{238, 240, 255}

\title{Portfolio\\
Développeur Web}
\author{Entreprise Webatom}
\date{09/2024}

\setlength{\footskip}{20pt}
\setlength\parindent{0pt}
\begin{document}
\renewcommand{\abstractname}{Présentation}
\maketitle
\begin{abstract}
    Je suis John Doe, développeur Web débutant dans l'entreprise Webatom. Actuellement en train de me former aux technologies du Web, j'apprends le métier de développeur full-stack.
\end{abstract}
\begin{figure}[H]
    \centering
    \includegraphics[scale=0.15]{./src/assets/atom.png}
\end{figure}
\clearpage
\tableofcontents
\newpage
\section{Résumé du projet}
Le projet consiste à réaliser un Portfolio comprenant toutes les créations faites auparavant. Il est réalisé en utilisant le framework Vue.JS. Le site Web consiste en la présentation sur une seule page des éléments suivants :\\
L'en-tête comprenant un menu adaptatif indiquant la section visible à l'écran ainsi qu'un bouton permettant de remonter en haut de la page. Une brève présentation de la personne dans une permière section. Une seconde section affichant une liste des précédents projets s'ouvrant dans un modal. Une troisième section incluant un formulaire de contact permettant d'envoyer le message directement par mail en utilisant EmailJS. Un bas de page personnalisé et identique sur toutes les pages. De plus, étant une page unique, un routeur est utilisé pour rediriger automatiquement toutes les autres URL vers une page d'erreur 404.

L'identité graphique étant libre, on opte pour une disposition épurée et des couleurs classiques. Le projet comporte une ou plusieurs vidéos pour montrer brièvement le comportement des anciens projets. Chaque modal permet ainsi de décrire le projet sélectionné, la date de création, le lien pour atteindre ce projet ainsi que les technologies utilisées.

Le projet étant consistant, il comporte 2 pages : la page d'accueil et la page d'erreur. Les composants définis seront le Header, le Footer, la fenêtre modale ainsi que le formulaire de contact. Un fichier regroupant les fonctions principales sera aussi créé.
\section{Conception du site}
Voici le lien Git-Hub pour récupérer le projet : \url{https://github.com/BushepII/portfolio.git}
\section{Validations W3C}
\end{document}